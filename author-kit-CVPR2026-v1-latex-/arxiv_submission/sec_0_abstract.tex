\begin{abstract}
Why do state-of-the-art OOD detection methods exhibit catastrophic failure when models are trained on single-domain datasets? We provide the first theoretical explanation for this phenomenon through the lens of information theory. We prove that supervised learning on single-domain data inevitably produces \emph{domain feature collapse} -- representations where $I(\rvx_\rvd; \rvz) = 0$, meaning domain-specific information is completely discarded. This is a fundamental consequence of information bottleneck optimization: models trained on single domains (e.g., medical images) learn to rely solely on class-specific features while discarding domain features, leading to catastrophic failure when detecting out-of-domain samples (e.g., achieving only 53\% FPR@95 on MNIST). We extend our analysis using Fano's inequality to quantify partial collapse in practical scenarios. To validate our theory, we introduce Domain Bench, a benchmark of single-domain datasets, and demonstrate that preserving $I(\rvx_\rvd; \rvz) > 0$ through domain filtering (using pretrained representations) resolves the failure mode. While domain filtering itself is conceptually straightforward, its effectiveness provides strong empirical evidence for our information-theoretic framework. Our work explains a puzzling empirical phenomenon, reveals fundamental limitations of supervised learning in narrow domains, and has broader implications for transfer learning and when to fine-tune versus freeze pretrained models.
\end{abstract}