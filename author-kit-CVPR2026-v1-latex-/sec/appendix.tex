\appendix

\begin{center}
    {\bf \LARGE Appendix}
\end{center}

\section{Properties of Mutual Information and Entropy}
\label{properties}
In this Section we enumerate some of the properties of mutual information that are used to prove the theorems reported in this work, initially proposed by \cite{shannon1948mathematical}. For any random variables $\rvw, \rvx, \rvy$ and $\rvz$ :

$\left(P_1\right)$ Positivity:
$$
I(\rvx ; \rvy) \geq 0, I(\rvx ; \rvy \mid \rvz) \geq 0
$$
$\left(P_2\right)$ Chain rule:
$$
I(\rvx \rvy ; \rvz)=I(\rvy ; \rvz)+I(\rvx ; \rvz \mid \rvy)
$$
$\left(P_3\right)$ Chain rule (Multivariate Mutual Information):
$$
I(\rvx ; \rvy ; \rvz)=I(\rvy ; \rvz)-I(\rvy ; \rvz \mid \rvx)
$$
$\left(P_4\right)$ Positivity of discrete entropy:
For discrete $\rvx$
$$
H(\rvx) \geq 0, H(\rvx \mid \rvy) \geq 0
$$
$\left(P_5\right)$ Entropy and Mutual Information
$$
H(\rvx)=H(\rvx \mid \rvy)+I(\rvx ; \rvy)
$$

$(P_6)$ Conditioning a variable cannot increase its entropy

$$
H(\rvy|\rvz) \leq H(\rvy)
$$

$(P_7)$ A variable knows about itself as much as any other variable can 

$$
I(\rvx;\rvx) \geq I(\rvx;\rvy) 
$$

$(P_8)$ Symmetry of Mutual Information

$$
I(\rvx;\rvy) = I(\rvy;\rvx) 
$$

$(P_9)$ Entropy and Conditional Mutual Information (This is simply $P_5$ conditioned on $\rvz$)

$$
I(\rvx;\rvy|\rvz)  = H(\rvx|\rvz) - H(\rvx|\rvy\rvz)
$$

$(P_{10})$ Functions of Independent Variables Remain Independent

$$
I(\rvx; \rvy) = 0 \rightarrow I(f(\rvx);\rvy) = 0
$$

\section{Fano's Inequality}
\label{fano}

\textcolor{black}{
\begin{theorem}
Fano's Inequality (See \citep{robert1952fano}). \\Let $\rvy$ be a discrete random variable representing the true label with $\mathcal{Y} $ possible values and cardinality of $|\mathcal{Y}| $ and $\rvx$ be a random variable used to predict $\rvy$. Let $e$ be the occurrence of an error such that $\rvy \neq \hat{ \rvy}$ where $\hat{ \rvy} = f(\rvx)$. Let $H_b$ represent the binary entropy function such that $H_b(e)=-P(e) \log P(e)-(1-P(e)) \log (1-P(e))$. The lower bound for $P(e)$ increases with lower mutual information $I(\rvx;\rvy)$.
\begin{align}
H_b(e) + P(e) \log(| \mathcal{Y} |-1) \geq H(\rvy) - I(\rvx; \rvy).
\end{align}
\end{theorem}}

Fano's inequality establishes a fundamental relationship between mutual information and prediction error. In the context of domain feature collapse, this theorem allows us to quantify the impact of partial compression scenarios where $I(\rvx_\rvd; \rvz) > 0$ but is still small. Even when domain features are not completely eliminated from the learned representation, Fano's inequality shows that low mutual information between domain features and the representation will result in high error rates for domain-based predictions, making OOD detection unreliable.

\subsection{Interpreting Experimental Results as Lower Bounds on Mutual Information}

Fano's inequality provides a crucial interpretation of our experimental results: OOD detection performance establishes a lower bound on $I(\rvx_\rvd; \rvz)$. Specifically, if a representation $\rvz$ achieves a certain level of performance in separating in-domain from out-of-domain samples, then $I(\rvx_\rvd; \rvz)$ must be at least high enough to support that performance level. By Fano's inequality, if the mutual information were lower, the error rate would necessarily increase, degrading OOD detection performance.

This allows us to indirectly measure mutual information through empirical performance: high separation ability (low FPR@95, high AUROC) implies a high lower bound on $I(\rvx_\rvd; \rvz)$, while poor separation ability implies a low lower bound. Our experiments thus not only demonstrate domain feature collapse qualitatively but also quantify the information-theoretic gap between supervised and pretrained representations.

\section{Main Theorems and Proofs}


\label{mainproofs}

We ignore cases where the determined variable has an entropy of 0. Generally, if $H(\rvy| \rvx) = 0 \rightarrow H(\rvy) > 0$. Also, we only consider cases where the random variables have more than zero entropy.

Note that $R_{\rvx}$ represents the support of random variable $\rvx$ such that $R_{\rvx} = \{\vx \in \R : P(\vx) > 0\}$.



\subsection{Lower Bound of Mutual Information for Sufficiency}
\begin{lemma}
    Let $\rvx$ and $\rvy$ be random variables with joint distribution $p(\rvx, \rvy)$. Let $\rvz$ be a representation of $\rvx$ that is sufficient, as per definition \ref{definesuff}. Then $I(\rvx;\rvz) \geq I(\rvz;\rvy)$ and $I(\rvx;\rvz) \geq I(\rvx;\rvy)$. 

    Hypothesis: 

    $(H_1)$ $\rvz$ is a representation of $\rvx: I(\rvy ; \rvz \mid \rvx)=0$
    
    $(H_2)$ $ \rvz$ is a sufficient representation of $\rvx: I(\rvx; \rvy| \rvz))=0$
    
    Thesis: 
    
    $(T_1)$ $\forall \rvz.I(\rvx;\rvz) \geq I(\rvz;\rvy), I(\rvx;\rvz) \geq I(\rvx;\rvy)$

    \begin{proof}By Construction 

    $$
    \begin{aligned}
    I(\rvx \rvy|\rvz)) &\stackrel{\left(H_2\right)}{=} 0 \\
    &\stackrel{\left(P_2\right)}{=} I(\rvz \rvy; \rvx) - I(\rvz; \rvx) \\
    &\stackrel{\left(P_2\right)}{=} I(\rvx; \rvy) + I(\rvx; \rvz|\rvy) - I(\rvz; \rvx) \\
    &\stackrel{\left(PropB1\right)}{=} I(\rvz; \rvy) + I(\rvx; \rvz|\rvy) - I(\rvz;\rvx) \\
    I(\rvz;\rvx) &= I(\rvz;\rvy) + I(\rvx; \rvz|\rvy)\\
   I(\rvz;\rvx) &\stackrel{\left(P_1\right)}{\geq} I(\rvz; \rvy)\\
    \end{aligned}
    $$

    Note that $I(\rvz; \rvy) = I(\rvx; \rvy)$ for all sufficient representations, as per proposition \ref{sufficiency}.

    This supports our intuition that the information in the representation consists of relevant information $I(\rvz;\rvy)$ and irrelevant information $I(\rvx; \rvz| \rvy)$. By definition of sufficiency, there must be enough information for $I(\rvz; \rvy)$ in $I(\rvx; \rvz)$, which is to say that the size of the encoding cannot be smaller than the minimum size to encode all of $I(\rvx; \rvy)$. 
    
    \end{proof} 
    \label{infobound}
\end{lemma}

\subsection{Factorization of Bottleneck Loss}

\begin{lemma}
    Let $\rvx$ be a random variable with label $\rvy$ such that $H(\rvy| \rvx) = 0$ and $\rvz$ is a sufficient representation of $\rvx$ for $\rvy$. The loss function 
    $\mathcal{L}=I(\rvx; \rvz)-\beta I(\rvz; \rvy)$ is equivalent to $\mathcal{L}=H(\rvz)-\beta I(\rvz; \rvy)$, with $\beta$ as some constant. 

    Hypothesis: 
    
    $(H_1)$  $\rvz$ is fully determined by $\rvx$ : $H(\rvz|\rvx) = 0$

    Thesis: 

    $(T_1)$ $ I(\rvx; \rvz)-\beta I(\rvz; \rvy) = H(\rvz)-\beta I(\rvz; \rvy)$

    \begin{proof} By Construction. 
    $$
        \begin{aligned}
        I(\rvx; \rvz)-\beta I(\rvz; \rvy) &\stackrel{\left(P_5\right)}{=} H(\rvz) - H(\rvz| \rvx) -\beta I(\rvz;\rvy) \\
        &\stackrel{\left(H_1\right)}{=} H(\rvz) -\beta I(\rvz; \rvy)
        \end{aligned}
    $$

    Due to the relationship between $\rvx$ and $\rvz$, we can create an intuitive factorization of the bottleneck loss function. Effectively, we want to maximize $I(\rvz; \rvy)$ while minimizing the information content of $\rvz$

    \end{proof}

    \label{lossfact}
\end{lemma}

\subsection{Conditional Mutual Information of Noise}

\begin{lemma}
Let $\rvx$ and $\rvy$ be independent random variables and $\rvz$ be a function of $\rvx$ with joint distribution $p(\rvx, \rvy, \rvz)$. The conditional mutual information $I(\rvx; \rvz| \rvy)$ is always equal to the mutual information $I(\rvx; \rvz)$. As in the information content is unchanged when adding noise. 

Hypothesis:

$(H_1)$ Independence of $\rvx$ and $\rvy$ : $I(\rvx;\rvy) = 0$

$(H_2)$  $\rvz$ is fully determined by $\rvx$ : $H(\rvz|\rvx) = 0$

Thesis: 

$(T_1)$ $I(\rvx; \rvz| \rvy) = I(\rvx; \rvz)$ 

\begin{proof}
    By Construction. 

    $(C_1)$ Demonstrates that $H(\rvz| \rvx \rvy) = 0$

    $$
\begin{aligned}
0 \stackrel{\left(P_4\right)}{\leq} H(\rvz|\rvx \rvy)&\stackrel{\left(P_6\right)}{\leq}H(\rvz|\rvx)\\
H(\rvz|\rvx \rvy)&\stackrel{\left(H_2\right)}{\leq} 0
\end{aligned}
$$

    $(C_2)$ Demonstrates that $I(\rvz;\rvy) = 0$ 
    $$
\begin{aligned}
I(\rvz; \rvy)&\stackrel{\left(H_2\right)}{=}I(f(\rvx);\rvy)\\
&\stackrel{\left(P_{10}\right)}{=}I(\rvx ; \rvy)\\
&\stackrel{\left(H_{1}\right)}{=}0\\
\end{aligned}
$$
    
    
Thus 

    $$
\begin{aligned}
I(\rvx; \rvz| \rvy) &\stackrel{\left(P_9\right)}{=} H(\rvz|\rvy) - H(\rvz|\rvx \rvy)\\
&\stackrel{\left(C_1\right)}{=} H(\rvz| \rvy)- 0\\
&\stackrel{\left(P_5\right)}{=} H(\rvz) - I(\rvz; \rvy) \\
&\stackrel{\left(C_2\right)}{=}H(\rvz) - 0 \\
&\stackrel{\left(H_2\right)}{=}H(\rvz) - H(\rvz| \rvx) \\
&\stackrel{\left(P_5\right)}{=}I(\rvx; \rvz)
\end{aligned}
$$
    
This supports the intuition that if one added a random noise channel it will not change the mutual information. 

\end{proof}
\label{infonoise}
\end{lemma}

\subsection{Domain Feature Collapse}

\begin{theorem}

Let $\rvx$ come from a distribution. $\rvx$ is composed of two independent variables $\rvx_\rvd$ and $\rvx_\rvy$, where $\rvx_\rvd$ is a set of domain features as per definition \ref{definedomainfeatures}. Let $\rvd$ be a domain label random variable generated from the labeling function $f_\rvd(\rvx_\rvd)$. In the single-domain training setting, $f_\rvd(\rvx_\rvd) = \rvd_1$ for all $\rvx$ in the training set, where $\rvd_1$ is a constant domain value. Let $\rvy$ be a class label generated from $f_\rvy(\rvx_\rvd, \rvx_\rvy) = \rvy$. Let $\rvz$ be any sufficient representation of $\rvx$ for $\rvy$ that satisfies the sufficiency definition \ref{definesuff} and minimizes the loss function $\mathcal{L} = I(\rvx_\rvd \rvx_\rvy; \rvz) - \beta I(\rvz;\rvy)$. The possible $\rvz$ that minimizes $\mathcal{L}$  and is sufficient must meet the condition $I(\rvx_\rvd; \rvz) = 0$.



Hypothesis:

$(H_1)$  $\rvz$ is fully determined by $\rvx$ : $H(\rvz|\rvx) = 0$

$(H_2)$  $\rvz$ is a representation of $\rvx: I(\rvy ; \rvz \mid \rvx)=0$

$(H_3)$  $\rvz$ is a sufficient representation of $\rvx: I(\rvx ; \rvy | \rvz)=0$

$(H_4)$ $\rvx$ is composed of two independent variables $\rvx_\rvd, \rvx_\rvy$ : $\rvx = \rvx_\rvd, \rvx_\rvy, I(\rvx_\rvy; \rvx_\rvd) = 0$

$(H_5)$ $\rvy$  and $\rvd$ are fully determined by $\rvx_\rvy$ and $\rvx_\rvd$, respectively: $H(\rvy | \rvx_\rvy) = 0$, $H(\rvd | \rvx_\rvd) = 0$. In the single-domain setting, since $\rvd = \rvd_1$ (constant) for all training samples, $I(\rvx_\rvd; \rvy) = 0$.

Thesis:

$(T_1)$ $\forall \rvz.I(\rvx_\rvd,\rvz) = 0$

\begin{proof} By Construction

$(C_1)$ demonstrates that $\mathcal{L}=H(\rvz) -\beta I(\rvz;\rvy)$ via factoring $I(\rvx_\rvd \rvx_\rvy;\rvz)$. Alternatively, Theorem \ref{lossfact} creates the same result. 

$$
    \begin{aligned}
    I(\rvx_\rvd \rvx_\rvy; \rvz) &\stackrel{\left(P_2\right)}{=} I(\rvx_\rvy; \rvz) + I(\rvx_\rvd; \rvz| \rvx_\rvy)\\
    &\stackrel{\left(P_5\right)}{=} H(\rvz) -  H(\rvz| \rvx_\rvy) + I(\rvx_\rvd; \rvz| \rvx_\rvy) \\
    &\stackrel{\left(P_9\right)}{=} H(\rvz) -  H(\rvz| \rvx_\rvy) + H(\rvz| \rvx_\rvy) - H(\rvz| \rvx_\rvy \rvx_\rvd) \\
    &\stackrel{\left(H_1\right)}{=}H(\rvz) -  H(\rvz| \rvx_\rvy) + H(\rvz| \rvx_\rvy) - 0 \\
    &\mathcal{L}=H(\rvz)  -\beta I(\rvz; \rvy)
    \end{aligned}
$$

$(C_2)$ Demonstrates that $I(\rvz; \rvy) = I(\rvx; \rvy)$ as per Theorem \ref{sufficiency}.

$(C_3)$ Demonstrates that $I(\rvz; \rvy)$ is a constant across all sufficient representations because Theorem \ref{sufficiency} applies.

$(C_4)$ Demonstrates that for all possible $\rvz$ satisfying $(H_3)$, their loss can be compared using only $\mathcal{L}_z = H(\rvz)$ for comparing across $\rvz$

$$
\begin{aligned}
    \frac{d\mathcal{L}}{d\rvz}  &\stackrel{\left(C_1\right)}{=}\frac{H(\rvz)}{d\rvz} - \frac{\beta I(\rvz; \rvy)}{d\rvz} \\
&\stackrel{\left(C_3\right)}{=}\frac{H(\rvz)}{d\rvz} - 0
\end{aligned}
$$

$(C_5)$ Demonstrates that the value of $H(\rvz)$ at all possible $\rvz$ that minimizes $\mathcal{L}$ is the same. Even for different minimal $\rvz$, they must have the same $H(\rvz)$ to all be minimal. When comparing possible minimal solutions to $\mathcal{L}$, $H(\rvz)$ is constant across all minimal solutions.

$(C_6)$ Demonstrates that any $\rvz$ that satisfies sufficiency must satisfy $I(\rvz; \rvx) \geq I(\rvz; \rvy)$ and $I(\rvz; \rvx) \geq I(\rvx; \rvy)$ as per Theorem \ref{infobound}.

$(C_7)$ Demonstrates that minima(s) exists only where $H(\rvz) = I(\rvz; \rvy)$ and $H(\rvz| \rvx) = 0$. Note that $H(\rvz) = I(\rvx; \rvy) = I(\rvz; \rvy)$ is the most compact representation size that is sufficient.

$$
\begin{aligned}
    I(\rvz;\rvx) &\stackrel{\left(C_6\right)}{\geq} I(\rvz; \rvy) \\
    H(\rvz) - H(\rvz| \rvx)  &\stackrel{\left(P_5\right)}{\geq} I(\rvz; \rvy)\\
    \forall \rvz\mid C_6 \land H_3 &\land I(\rvz; \rvx) > I(\rvz; \rvy) . \\
    \exists \rvz' \mid \rvz' = f(\rvz) &\land I(\rvz; \rvx) > I(\rvz'; \rvx) \land C_6 \land H_3
\end{aligned}
$$
From $(C_7)$ there exists only 3 types of minimas, separated by their dependence on the variables  $\rvx_\rvy, \rvx_\rvd$. As per $(H_1)$, any $\rvz$ must follow one of the 3 types.

\begin{enumerate}
    \item Dependent only on $\rvx_\rvy$: $\forall \rvz|H(\rvz|\rvx_\rvy)=0 \rightarrow I(\rvx_\rvd; \rvz) = 0$

    \item Dependent only on $\rvx_\rvd$: $\forall \rvz|H(\rvz| \rvx_\rvd)=0 \rightarrow I(\rvx_\rvd; \rvz) > 0$

    \item Dependent on both $\rvx_\rvy \rvx_\rvd$: $\forall \rvz|H(\rvz|\rvx_\rvy, \rvx_\rvd)=0 \land h(\rvz| \rvx_\rvy)>0 \land H(\rvz| \rvx_\rvd)>0 \rightarrow I(\rvx_\rvd; \rvz) > 0$
\end{enumerate}

From here we will show that all type 2 and type 3 minimas always fail $(H_3)$ or have greater $\mathcal{L}$ than any type 1 minima.

\textbf{Type 1} $\rvx_\rvy$: $\forall \rvz|H(\rvz| \rvx_\rvy)=0 \rightarrow I(\rvx_\rvd; \rvz) = 0$

$(C_8)$ Demonstrates that there exists $H(\rvz) = I(\rvz; \rvy) = I(\rvx_\rvy; \rvz)$ and it is a set of minimas satisfying $(C_7)$. This also establishes an upper bound for solutions to $\mathcal{L}$ due to $(C_5)$. Therefore, any solution for type 1, type 2, and type 3 must satisfy $I(\rvz; \rvy) \leq I(\rvx_\rvy; \rvz)$ to be sufficient and $I(\rvz; \rvy) = I(\rvx_\rvy; \rvz)$ to be minimal.

$$
\begin{aligned}
    I(\rvz; \rvy)&\stackrel{\left(C_6\right)}{\leq} I(\rvz; \rvx)  \\
    &\stackrel{\left(H_4\right)}{\leq}  I(\rvx_\rvy, \rvx_\rvd; \rvz) \\
    &\stackrel{\left(P_2\right)}{\leq} I(\rvx_\rvd; \rvz) + I(\rvx_\rvy; \rvz| \rvx_\rvd)\\
    &\stackrel{\left(Type1\right)}{\leq} 0 + I(\rvx_\rvy; \rvz| \rvx_\rvd)\\
    &\stackrel{\left(Theorum\ref{infonoise}\right)}{\leq} I(\rvx_\rvy; \rvz)\\
    &\stackrel{\left(P_5\right)}{\leq} H(\rvz) - H(\rvz| \rvx_\rvy)\\
    \exists \rvz &|I(\rvx_\rvy; \rvz)  = I(\rvz; \rvy) = I(\rvx; \rvz) = I(\rvx; \rvy)
\end{aligned}
$$

$(C_{9})$ Demonstrates that there exists no $H(\rvz') < H(\rvz)$ that satisfies sufficiency if $\rvz$ satisfies $(C_8)$ and is also $I(\rvz; \rvx_\rvd) = 0$.

$$
\begin{aligned}
    C_8 &\rightarrow  I(\rvx_\rvy; \rvz) = I(\rvx; \rvy)\\
    H(\rvz') < H(\rvz) & \rightarrow I(\rvx_\rvy; \rvz') < I(\rvx_\rvy; \rvz) \\
    \rightarrow \neg (C_2) &: I(\rvx_\rvy; \rvz') < I(\rvx_\rvy; \rvz) = I(\rvy; \rvz) = I(\rvx; \rvy)
\end{aligned}
$$

\textbf{Type 2} $\rvx_\rvd$: $\forall \rvz|H(\rvz| \rvx_\rvd)=0 \rightarrow I(\rvx_\rvd; \rvz) > 0$

$(C_{10})$ Demonstrates that no type 2 minima can exist, simply because it would contain no information regarding $\rvx_\rvy$, thus failing to satisfy $(H_3)$. This is because $\rvz$ cannot contain any information about $\rvx_\rvy$, otherwise we would not satisfy $H(\rvz|\rvx_\rvd)=0$. If the representation $\rvz$ contains no information about $\rvy$, then it is not sufficient.

$$
\begin{aligned}
    H(\rvz| \rvx_\rvd) = 0 & \rightarrow \rvz = f(\rvx_\rvd) \\
    0 &\stackrel{\left(H_4\right)}{=} I(\rvx_\rvy; \rvx_\rvd)\\
    &\stackrel{\left(P_{10}\right)}{=} I(f(\rvx_\rvy);\rvx_\rvd)\\
    &\stackrel{\left(H_5\right)}{=} I(\rvy; \rvx_\rvd)\\
    &\stackrel{\left(P_{10}\right)}{=} I(\rvy;f(\rvx_\rvd))\\
    0&=I(\rvy;\rvz)
\end{aligned}
$$

\textbf{Type 3} $\rvx_\rvy, \rvx_\rvd$: $\forall \rvz|H(\rvz| \rvx_\rvy , \rvx_\rvd)=0 \land H(\rvz|\rvx_\rvy)>0 \land H(\rvz| \rvx_\rvd)>0 \rightarrow I(\rvx_\rvd; \rvz) > 0$

$(C_{11})$ Demonstrates that any $\rvz$ that could be minimal must also satisfy $(C_8)$ for sufficiency. Note that $(C_8)$ implies that any $I(\rvx_\rvy; \rvz) >  I(\rvz;\rvy)$ is not minimal.

$$
\begin{aligned}
    I(\rvz; \rvy)&\stackrel{\left(C_6\right)}{\leq} I(\rvz; \rvx)  \\
    &\stackrel{\left(H_4\right)}{\leq}  I(\rvx_\rvy \rvx_\rvd; \rvz) \\
    I(\rvz; \rvy) &\stackrel{\left(P_2\right)}{\leq} I(\rvx_\rvy; \rvz) + I(\rvx_\rvd; \rvz| \rvx_\rvy) \\
    & (C_8) \rightarrow I(\rvx_\rvy; \rvz) =  I(\rvz; \rvy) \\
\end{aligned}
$$

$(C_{12})$ Demonstrates that any $\rvz'$ where $I(\rvz'; \rvx_\rvd)>I(\rvz; \rvx_\rvd)$ and $I(\rvz; \rvx_\rvd) = 0$ that maintains $H(\rvz') = H(\rvz)$ results in  solutions that are not sufficient as required by $(H_3)$ because we know that the size of the representation must be at least $I(\rvx; \rvy)$ as defined in $(C_6)$.

$$
\begin{aligned}
    C_8 &\rightarrow  H(\rvz) \text{ is constant across all minima}\\
    C_8 &\rightarrow  H(\rvz) = H(\rvz')\text{ for } \rvz' \text{ to be minimal}\\
    C_8 &\rightarrow  I(\rvx_\rvy; \rvz) = I(\rvx; \rvy)\\
    I(\rvx_\rvd; \rvz) = 0 &\rightarrow H(\rvz| \rvx_\rvy) = 0 \\
    \forall \rvz'|I(\rvx_\rvd; \rvz') > 0  &:  H(\rvz'| \rvx_\rvy) > H(\rvz| \rvx_\rvy) \\
   H(\rvz'| \rvx_\rvy) &> H(\rvz| \rvx_\rvy) \\
   \rightarrow H(\rvz') - H(\rvz'| \rvx_\rvy) &< H(\rvz) - H(\rvz| \rvx_\rvy)  \\
    &\stackrel{\left(P_5\right)}{\rightarrow}  I(\rvx_\rvy; \rvz') < I(\rvx_\rvy; \rvz)  \\
    \rightarrow \neg (C_6) &:  I(\rvx_\rvy; \rvz') < I(\rvx; \rvy)
\end{aligned}
$$

$(C_{13})$ Demonstrates that combining $(C_{11})$ and $(C_{12})$, there is no type 3 solution that has an equal $\mathcal{L}$ to the minimal type 1 solution that also maintains sufficiency $(H_3)$ and $(C_6)$. This confirms the definition of entropy, in that encoding more independent information requires more bits or nats.

This means that only a type 1 solution can be both minimal and sufficient, which proves the thesis.

To summarize this proof, we can compare the losses of all sufficient solutions with $\mathcal{L} = H(\rvz)$. Of those sufficient solutions, the one that minimizes $\mathcal{L}$ is the one with the smallest $H(\rvz)$. The minimal sufficient representation is $\rvz$ that captures only all of $I(\rvx_\rvy; \rvy)$ and nothing else. Thus the minimal $\rvz$ cannot have $I(\rvx_\rvd; \rvz) > 0$ because such $\rvz$ would encode information outside of $I(\rvx_\rvy; \rvy)$.

% that previously satisfied $(C_8)$ will have a greater loss if $I(\mathbf{x_2;z}) > 0$.


\end{proof}
    \label{genloss}



\end{theorem}

\section{Theorems and Proofs of Previous Work}

This Section contains the supporting theorems and proofs provided by previous work \citep{federici2020learning}.

When random variable $\rvz$ is defined to be a representation of another random variable $\rvx$, we state that $\rvz$ is conditionally independent from any other variable in the system once $\rvx$ is observed. This does not imply that $\rvz$ must be a deterministic function of $\rvx$, but that the source of stochasticity for $\rvz$ is independent of the other random variables. As a result whenever $\rvz$ is a representation of $\rvx$ :
$$
I(\rvz ; \rva \mid \rvx \rvb)=0,
$$
for any variable (or groups of variables) $\rva$ and $\rvb$ in the system. This condition is accounts for the randomness experienced in training neural networks and the error expected from human labelers. This condition applies to this and the following sections.

\subsection{Sufficiency}

\begin{proposition}

Let $\rvx$ and $\rvy$ be random variables from joint distribution $p(\rvx, \rvy)$. Let $\rvz$ be a representation of $\rvx$, then $\rvz$ is sufficient for $\rvy$ if and only if $I(\rvx ; \rvy)=I(\rvy ; \rvz)$

Hypothesis:

$\left(H_1\right) \rvz$ is a representation of $\rvx: I(\rvy ; \rvz \mid \rvx)=0$

Thesis:

$\left(T_1\right) I(\rvx ; \rvy \mid \rvz)=0 \Longleftrightarrow I(\rvx ; \rvy)=I(\rvy ; \rvz)$

\begin{proof}
$$
\begin{aligned}
I(\rvx ; \rvy \mid \rvz) & \stackrel{\left(P_3\right)}{=} I(\rvx ; \rvy)-I(\rvx ; \rvy ; \rvz) \\
I(\rvx ; \rvy)-I(\rvx ; \rvy ; \rvz) &
\stackrel{\left(P_3\right)}{=} I(\rvx ; \rvy)-I(\rvy ; \rvz)+I(\rvy ; \rvz \mid \rvx) \\
& \stackrel{\left(H_1\right)}{=} I(\rvx ; \rvy)-I(\rvy ; \rvz)
\end{aligned}
$$

Since both $I(\rvx ; \rvy)$ and $I(\rvy ; \rvz)$ are non-negative $\left(P_1\right), I(\rvx ; \rvy \mid \rvz)=0 \Longleftrightarrow I(\rvy ; \rvz)=I(\rvx ; \rvy)$

\end{proof}
\label{sufficiency}
\end{proposition}

\section{Two Stage Domain Filter}

\label{twostageapp}

\begin{algorithm}[H]
\caption{Two-Stage Domain Filter for OOD Detection}
\label{twostage}
\begin{algorithmic}[1]
\State \textbf{Input:}
    \State $\rvx$: Input sample
    \State $\sX_{train}$: Training dataset
    \State $k$: Number of neighbors (default=50)
    \State $\rvt_\rvd$: Domain threshold (99th percentile)

\State \textbf{Output:}
    \State OOD decision $\in \{\text{True}, \text{False}\}$

\Procedure{DomainFilter}{$\rvx, \sX_{train}, k, \rvt_\rvd$}
    \State $d_k \gets \text{KNN-Distance}(\rvx, \sX_{train}, k)$ \Comment{$k^{th}$ neighbor distance}
    \If{$d_k > \rvt_\rvd$}
        \State \Return \text{True} \Comment{Out-of-Domain}
    \Else
        \State \Return \text{False} \Comment{In-Domain}
    \EndIf
\EndProcedure

\Procedure{TwoStageDetection}{$x$}
    \State \text{// Stage 1: Domain Filtering}
    \If{\Call{DomainFilter}{$\rvx, \sX_{train}, k, \rvt_\rvd$}}
        \State \Return \text{True} \Comment{Reject as OOD (Avoids Domain Feature Collapse)}
    \EndIf

    \State \text{// Stage 2: In-Distribution OOD Detection}
    \State $s \gets \text{OOD-Score}(\rvx)$ \Comment{Using preferred OOD detector}
    \If{$s > \tau$} \Comment{$\tau$ is OOD threshold}
        \State \Return \text{True}
    \Else
        \State \Return \text{False}
    \EndIf
\EndProcedure

\State \textbf{Threshold Calibration:}
\State $\rvt_\rvd \gets \text{Percentile}(\{f_{knn}(\rvx_i) | \rvx_i \in \sX_{train}\}, 99\%)$

\end{algorithmic}
\end{algorithm}


\section{Detailed Experimental Setup}

\label{setup}

\subsection{Adjacent OOD Construction}

For each seed, we randomly select $1/3$ of ID classes to be treated as in domain OOD classes. This is repeated 5 times per dataset, such that all 3 training methods use the same 5 seeds for their experiments.

\subsection{Cross Entropy Resnet50}

We train the Cross Entropy Resnet50 using the baseline training pipeline from OpenOOD. This pipeline uses an SGD optimizer with an initial LR of 0.1, momentum of 0.9, and a weight decay of 0.0005. We use a cosine annealing schedule for the learning rate. We train with a 256 batch size and an image size of 64. We use the OpenOOD base preprocesser for augmentations, which only includes a center crop, horizontal flip, and random crop. The ResNet50 is initialized with the default Torchvision weights, derived from Imagenet.

The model with the best accuracy on the validation set is selected for OOD evaluation.

We use the OpenOOD OODEvaluator class to evaluate OOD performance. Hyper parameters are selected using the ID validation set and the Tiny ImageNet validation set. Hyperparameters are selected using the configurations provided by OpenOOD. We limit the domain filter's possible $k$ values to $[50, 100, 200]$.

\subsection{Cross Entropy DinoV2}

We train the Cross Entropy DinoV2 Vit-S14 using the baseline training pipeline from OpenOOD. We modify the pipeline to use an Adam optimizer with an initial LR of 0.00001 and a weight decay of 0.0005. We use a cosine annealing schedule for the learning rate. We train with a 128 batch size and an image size of 224. We use the OpenOOD base preprocesser for augmentations, which only includes a center crop, horizontal flip, and random crop.

The model with the best accuracy on the validation set is selected for OOD evaluation.

We use evaluation process as the Cross Entropy Resnet50.

\subsection{Supervised Contrastive Learning ResNet50}

We implement a Supervised Contrastive Learning pipeline in OpenOOD by following the implementation by \cite{sehwag2021ssd}. This pipeline uses an SGD optimizer with an initial LR of 0.5, momentum of 0.9, a weight decay of 0.0005, and a SimCLR temperature of 0.5. The model trains for 10 warm up epochs using a cyclic LR scheduler followed by 500 epoches using a cosine annealing LR scheduler. Preprocessing follows \citep{sehwag2021ssd}, where two augmented copies of an image are generated for contrastive learning, using RandomResizeCrop, RandomHorizontalFlip, ColorJitter, and GrayScale.

The model with the best accuracy on the validation set is selected for OOD evaluation, with accuracy established using a KNN fitted on the learned representations.

We use evaluation process as the Cross Entropy Resnet50, except all logit based OOD methods are not evaluated (due to the lack of a classification head).




\section{Detailed Experimental Results}


\label{expresult}

\subsection{OOD Method References}

\label{oodref}
PT KNN refers to a KNN OOD detector \citep{sun2022out} using only a pretrained DinoV2. DF + KNN refers to the two stage domain filter combined with an KNN OOD detector \citep{sun2022out} and likewise with DF + ReAct \citep{sun2021react}. Other listed methods are MSP \citep{hendrycks2016baseline}, Energy \citep{liu2020energy}, Mahalanobis \citep{lee2018simple}, NCI \citep{liu2025detecting}, and KNN \citep{sun2022out}.



\subsection{Experimental Results by ID Dataset and OOD Method }

We provide FPR@95 and AUROC scores for each ID dataset and OOD detection method, across the 3 models. These results can be found in tables \ref{tab:fpr95_results_resnet},  \ref{tab:auroc_results_resnet}, \ref{tab:fpr95_results_dino}, \ref{tab:auroc_results_dino}, \ref{tab:fpr95_results_simclr}, and \ref{tab:auroc_results_simclr}.

\begin{sidewaystable}
\caption{FPR@95 Performance by Method and ID Dataset For Supervised Cross Entropy Trained Resnet50}

\label{tab:fpr95_results_resnet}
\begin{tabular}{llllllllllll}
\toprule
  & Butterfly & Cards & Colon & Eurosat & Fashion & Food & Garbage & Plant & Rock & Tissue & Yoga \\
method &  &  &  &  &  &  &  &  &  &  &  \\
\midrule
PT KNN & 93.0 / 0.0 & 91.9 / 2.9 & 67.4 / 0.0 & 69.1 / 0.3 & 65.6 / 1.8 & 80.0 / 0.6 & 87.2 / 0.4 & 62.7 / 0.0 & 91.9 / 6.6 & 89.3 / 0.0 & 83.7 / 1.7 \\
MSP & 50.7 / 15.4 & 57.3 / 23.0 & 59.1 / 53.0 & 41.3 / 49.8 & 75.6 / 19.7 & 74.9 / 63.7 & 68.0 / 42.0 & 30.1 / 8.6 & 85.8 / 71.8 & 84.2 / 76.6 & 77.1 / 36.9 \\
Energy & 60.8 / 13.5 & 64.7 / 19.9 & 61.0 / 70.7 & 42.5 / 50.1 & 77.2 / 15.8 & 75.2 / 62.9 & 78.7 / 54.3 & 31.5 / 11.3 & 86.7 / 71.2 & 84.4 / 79.2 & 76.7 / 36.4 \\
Mahalanobis & 57.6 / 6.7 & 45.5 / 15.1 & 40.8 / 12.5 & 51.4 / 13.7 & 75.8 / 6.3 & 76.8 / 52.1 & 59.8 / 13.9 & 27.1 / 1.0 & 83.1 / 44.2 & 91.4 / 3.8 & 72.3 / 36.9 \\
NCI & 75.7 / 10.7 & 74.9 / 14.3 & 74.5 / 24.8 & 72.7 / 57.1 & 65.9 / 38.6 & 80.4 / 65.4 & 74.2 / 31.4 & 58.7 / 48.0 & 75.9 / 64.0 & 84.5 / 35.7 & 83.3 / 34.8 \\
KNN & 67.0 / 20.5 & 52.3 / 13.8 & 40.0 / 13.2 & 48.4 / 31.3 & 79.0 / 14.9 & 73.3 / 62.7 & 77.9 / 33.3 & 29.6 / 2.4 & 77.3 / 61.8 & 92.6 / 31.6 & 83.3 / 33.7 \\
ReAct & 81.7 / 40.5 & 71.2 / 21.7 & 39.0 / 61.2 & 55.5 / 54.4 & 82.0 / 37.4 & 85.9 / 71.1 & 82.9 / 58.6 & 63.0 / 44.4 & 84.7 / 75.0 & 81.7 / 48.0 & 75.8 / 39.2 \\
DF + KNN & 66.5 / 0.5 & 52.3 / 5.5 & 41.5 / 0.2 & 49.6 / 1.5 & 77.5 / 1.4 & 73.5 / 2.3 & 76.5 / 2.1 & 29.6 / 0.1 & 75.1 / 52.5 & 92.2 / 0.4 & 84.0 / 11.4 \\
DF + ReAct & 82.2 / 0.3 & 70.5 / 6.2 & 40.6 / 0.7 & 65.2 / 4.3 & 71.0 / 1.6 & 86.4 / 2.2 & 82.9 / 1.8 & 62.7 / 0.3 & 84.9 / 61.0 & 81.9 / 0.7 & 79.4 / 22.8 \\
DF + Mahalanobis & 56.7 / 4.7 & 45.5 / 13.0 & 40.4 / 6.9 & 51.4 / 10.4 & 76.3 / 3.2 & 76.6 / 39.1 & 61.6 / 12.7 & 26.7 / 0.8 & 82.0 / 39.8 & 91.3 / 0.9 & 73.0 / 26.1 \\
\bottomrule
\end{tabular}

\end{sidewaystable}


\begin{sidewaystable}
\caption{AUROC Performance by Method and ID Dataset For Supervised Cross Entropy Trained Resnet 50}


\label{tab:auroc_results_resnet}
\begin{tabular}{llllllllllll}
\toprule
 & Butterfly & Cards & Colon & Eurosat & Fashion & Food & Garbage & Plant & Rock & Tissue & Yoga \\
method &  &  &  &  &  &  &  &  &  &  &  \\
\midrule
PT KNN & 53.0 / 99.9 & 52.8 / 98.8 & 73.0 / 99.9 & 73.4 / 99.9 & 78.4 / 99.2 & 67.5 / 99.6 & 62.4 / 99.9 & 81.4 / 99.9 & 58.2 / 98.0 & 58.7 / 99.9 & 57.8 / 99.3 \\
MSP & 86.3 / 94.9 & 73.7 / 92.2 & 85.0 / 90.9 & 89.3 / 89.1 & 73.8 / 93.4 & 66.4 / 74.1 & 76.9 / 84.4 & 95.3 / 98.1 & 64.2 / 66.2 & 64.2 / 71.3 & 71.8 / 85.5 \\
Energy & 86.2 / 96.4 & 73.9 / 94.0 & 84.7 / 86.7 & 89.6 / 89.6 & 73.2 / 96.0 & 66.9 / 75.5 & 73.7 / 81.8 & 95.1 / 97.8 & 64.0 / 68.6 & 65.5 / 71.5 & 71.3 / 86.6 \\
Mahalanobis & 86.8 / 98.5 & 78.3 / 94.7 & 89.3 / 96.3 & 88.4 / 97.3 & 75.3 / 98.5 & 66.8 / 84.2 & 77.5 / 96.1 & 96.0 / 99.8 & 63.7 / 85.1 & 58.0 / 99.2 & 68.0 / 79.8 \\
NCI & 83.7 / 96.9 & 70.3 / 95.8 & 68.3 / 95.1 & 80.1 / 84.2 & 76.0 / 90.6 & 65.7 / 73.3 & 72.7 / 90.0 & 83.5 / 87.1 & 67.4 / 71.9 & 65.2 / 84.8 & 67.7 / 86.8 \\
KNN & 81.5 / 94.1 & 76.2 / 93.7 & 89.4 / 96.8 & 88.3 / 92.9 & 75.1 / 96.3 & 66.6 / 74.2 & 65.2 / 88.4 & 95.4 / 99.3 & 65.0 / 73.8 & 59.2 / 89.2 & 64.2 / 86.1 \\
ReAct & 74.0 / 88.1 & 71.8 / 93.9 & 86.3 / 78.6 & 86.0 / 87.9 & 67.1 / 89.8 & 64.8 / 74.8 & 68.2 / 81.3 & 85.4 / 89.8 & 63.8 / 65.2 & 68.3 / 79.5 & 71.8 / 85.7 \\
DF + KNN & 80.6 / 99.8 & 76.0 / 98.1 & 88.6 / 99.9 & 87.8 / 99.7 & 75.4 / 99.6 & 66.8 / 99.1 & 64.4 / 99.5 & 94.6 / 99.9 & 64.2 / 84.5 & 59.4 / 99.9 & 65.0 / 96.7 \\
DF + ReAct & 73.8 / 99.9 & 72.0 / 98.6 & 84.7 / 99.8 & 79.9 / 99.3 & 71.7 / 99.6 & 64.4 / 99.1 & 68.6 / 99.5 & 85.6 / 99.9 & 63.1 / 80.3 & 68.6 / 99.7 & 69.0 / 95.2 \\
DF + Mahalanobis & 86.9 / 99.0 & 78.6 / 95.6 & 89.3 / 97.9 & 88.4 / 97.9 & 75.7 / 99.2 & 66.9 / 89.5 & 77.4 / 96.3 & 96.0 / 99.8 & 63.0 / 87.4 & 58.3 / 99.8 & 68.3 / 87.8 \\
\bottomrule
\end{tabular}

\end{sidewaystable}

\begin{sidewaystable}
\caption{FPR@95 Performance by Method and ID Dataset For Supervised Cross Entropy Trained DinoV2}


\label{tab:fpr95_results_dino}

\begin{tabular}{llllllllllll}
\toprule
 & Butterfly & Cards & Colon & Eurosat & Fashion & Food & Garbage & Plant & Rock & Tissue & Yoga \\
method &  &  &  &  &  &  &  &  &  &  &  \\
\midrule
PT KNN & 93.0 / 0.0 & 91.9 / 2.9 & 67.4 / 0.0 & 69.1 / 0.3 & 65.6 / 1.8 & 80.0 / 0.6 & 87.2 / 0.4 & 62.7 / 0.0 & 91.9 / 6.6 & 89.3 / 0.0 & 83.7 / 1.7 \\
MSP & 64.4 / 44.6 & 64.0 / 38.1 & 58.9 / 11.4 & 57.8 / 19.4 & 65.7 / 52.2 & 69.6 / 53.5 & 81.2 / 60.5 & 21.2 / 5.2 & 87.2 / 87.0 & 83.6 / 77.6 & 85.6 / 62.6 \\
Energy & 69.7 / 39.5 & 62.5 / 26.4 & 54.8 / 16.2 & 54.6 / 15.8 & 70.9 / 46.9 & 67.9 / 41.1 & 81.9 / 55.8 & 20.4 / 3.5 & 87.3 / 85.8 & 81.6 / 64.3 & 84.0 / 59.2 \\
Mahalanobis & 74.8 / 36.7 & 68.2 / 18.6 & 37.6 / 0.3 & 27.5 / 6.4 & 67.7 / 4.9 & 76.5 / 37.3 & 75.8 / 27.5 & 26.0 / 2.1 & 90.8 / 62.6 & 87.6 / 0.2 & 76.4 / 49.1 \\
NCI & 68.4 / 38.8 & 62.5 / 26.9 & 55.9 / 13.3 & 54.2 / 19.8 & 71.1 / 32.9 & 67.7 / 42.2 & 83.7 / 48.1 & 21.5 / 3.9 & 85.6 / 76.7 & 93.0 / 65.5 & 86.2 / 58.6 \\
KNN & 77.6 / 48.7 & 69.7 / 21.7 & 31.2 / 1.6 & 29.8 / 23.4 & 68.5 / 8.4 & 70.8 / 48.4 & 79.2 / 45.8 & 19.4 / 3.1 & 90.0 / 72.1 & 87.1 / 1.4 & 79.2 / 51.2 \\
ReAct & 69.4 / 39.7 & 64.2 / 26.2 & 48.9 / 19.6 & 51.2 / 15.8 & 72.4 / 40.5 & 68.1 / 40.3 & 82.1 / 55.2 & 19.8 / 3.4 & 87.3 / 84.5 & 81.4 / 61.7 & 81.8 / 57.6 \\
DF + KNN & 87.5 / 0.8 & 82.6 / 8.8 & 33.3 / 0.1 & 33.1 / 0.8 & 68.9 / 2.7 & 72.4 / 2.1 & 79.7 / 1.4 & 19.8 / 0.1 & 91.5 / 57.9 & 87.4 / 0.1 & 81.1 / 14.3 \\
DF + ReAct & 69.3 / 0.7 & 63.7 / 8.8 & 49.1 / 0.3 & 52.6 / 0.8 & 71.9 / 5.0 & 67.8 / 1.9 & 81.3 / 1.9 & 20.5 / 0.1 & 87.3 / 74.8 & 81.5 / 0.8 & 83.2 / 16.1 \\
DF + Mahalanobis & 75.2 / 0.7 & 67.5 / 7.1 & 37.9 / 0.0 & 29.8 / 0.8 & 67.4 / 2.5 & 76.4 / 2.1 & 76.1 / 1.5 & 25.8 / 0.1 & 86.8 / 50.5 & 87.7 / 0.0 & 76.9 / 14.1 \\
\bottomrule
\end{tabular}

\end{sidewaystable}


\begin{sidewaystable}
\caption{AUROC Performance by Method and ID Dataset For Supervised Cross Entropy Trained DinoV2}


\label{tab:auroc_results_dino}

\begin{tabular}{llllllllllll}
\toprule
  & Butterfly & Cards & Colon & Eurosat & Fashion & Food & Garbage & Plant & Rock & Tissue & Yoga \\
method &  &  &  &  &  &  &  &  &  &  &  \\
\midrule
PT KNN & 53.0 / 99.9 & 52.8 / 98.8 & 73.0 / 99.9 & 73.4 / 99.9 & 78.4 / 99.2 & 67.5 / 99.6 & 62.4 / 99.9 & 81.4 / 99.9 & 58.2 / 98.0 & 58.7 / 99.9 & 57.8 / 99.3 \\
MSP & 77.2 / 82.2 & 73.1 / 82.9 & 85.1 / 97.2 & 87.3 / 94.3 & 73.6 / 80.1 & 69.2 / 78.6 & 68.3 / 75.4 & 95.7 / 98.6 & 61.3 / 53.6 & 58.7 / 60.0 & 65.1 / 73.5 \\
Energy & 78.3 / 85.5 & 72.8 / 89.0 & 84.8 / 96.6 & 87.9 / 96.1 & 70.0 / 84.6 & 71.2 / 86.2 & 67.2 / 77.4 & 96.2 / 99.2 & 66.5 / 52.3 & 61.0 / 67.9 & 66.4 / 75.8 \\
Mahalanobis & 74.5 / 86.6 & 66.1 / 92.2 & 90.0 / 99.9 & 93.7 / 98.8 & 79.3 / 98.7 & 66.8 / 88.0 & 69.2 / 93.0 & 93.6 / 99.5 & 58.4 / 74.2 & 60.8 / 99.9 & 68.2 / 78.1 \\
NCI & 78.4 / 86.0 & 73.0 / 88.8 & 86.1 / 96.8 & 86.6 / 95.1 & 71.5 / 90.3 & 71.2 / 85.4 & 65.7 / 80.9 & 96.0 / 99.1 & 64.8 / 62.5 & 50.9 / 69.6 & 64.0 / 75.5 \\
KNN & 72.4 / 82.2 & 68.2 / 91.4 & 91.8 / 99.6 & 92.6 / 94.7 & 80.1 / 97.7 & 69.4 / 83.1 & 64.1 / 85.5 & 96.0 / 99.2 & 57.8 / 66.0 & 60.0 / 99.7 & 66.4 / 78.0 \\
ReAct & 78.3 / 85.4 & 72.6 / 89.1 & 86.4 / 95.0 & 88.5 / 96.2 & 72.2 / 87.6 & 71.2 / 86.8 & 66.7 / 78.1 & 96.2 / 99.2 & 66.9 / 53.9 & 61.8 / 71.2 & 67.3 / 76.5 \\
DF + KNN & 66.7 / 99.7 & 61.5 / 96.5 & 90.4 / 99.9 & 91.3 / 99.8 & 79.7 / 99.2 & 68.8 / 99.2 & 63.2 / 99.5 & 95.3 / 99.9 & 56.5 / 82.8 & 60.1 / 99.9 & 66.0 / 96.0 \\
DF + ReAct & 77.8 / 99.7 & 72.7 / 96.9 & 86.2 / 99.9 & 88.1 / 99.8 & 73.0 / 98.3 & 71.4 / 99.3 & 67.3 / 99.4 & 95.5 / 99.9 & 65.7 / 75.0 & 61.5 / 99.6 & 66.3 / 94.9 \\
DF + Mahalanobis & 74.1 / 99.6 & 66.1 / 97.4 & 89.7 / 99.9 & 93.0 / 99.8 & 79.4 / 99.3 & 67.1 / 99.3 & 69.7 / 99.6 & 93.2 / 99.9 & 60.1 / 82.9 & 60.9 / 99.9 & 67.9 / 95.4 \\
\bottomrule
\end{tabular}

\end{sidewaystable}



\begin{sidewaystable}
\caption{FPR@95 Performance by Method and ID Dataset For Supervised Constrastive Learning Trained Resnet50}

\label{tab:fpr95_results_simclr}
\begin{tabular}{llllllllllll}
\toprule
 & Butterfly & Cards & Colon & Eurosat & Fashion & Food & Garbage & Plant & Rock & Tissue & Yoga \\
method &  &  &  &  &  &  &  &  &  &  &  \\
\midrule
KNN & 72.0 / 38.1 & 80.9 / 23.5 & 28.1 / 21.8 & 34.1 / 20.6 & 59.6 / 23.2 & 65.7 / 44.0 & 84.6 / 70.2 & 41.2 / 3.7 & 88.3 / 84.0 & 90.2 / 61.8 & 70.9 / 36.1 \\
DF + KNN & 79.5 / 0.7 & 81.6 / 8.3 & 32.0 / 0.4 & 41.7 / 1.2 & 56.9 / 3.1 & 65.9 / 1.6 & 85.9 / 3.2 & 42.2 / 0.1 & 87.3 / 71.0 & 89.5 / 0.6 & 73.4 / 11.0 \\
Mahalanobis & 64.9 / 23.8 & 74.3 / 15.5 & 38.5 / 57.1 & 26.1 / 17.5 & 69.1 / 38.8 & 73.0 / 49.6 & 77.2 / 45.6 & 51.1 / 15.5 & 97.0 / 72.1 & 94.6 / 64.4 & 69.2 / 34.2 \\
DF + Mahalanobis & 64.7 / 21.6 & 73.2 / 15.0 & 37.1 / 1.0 & 25.1 / 4.2 & 70.2 / 5.9 & 72.6 / 12.5 & 77.0 / 29.0 & 48.1 / 0.4 & 96.5 / 64.6 & 91.5 / 2.2 & 66.9 / 24.2 \\
PT KNN & 93.0 / 0.0 & 91.9 / 2.9 & 67.4 / 0.0 & 69.1 / 0.3 & 65.6 / 1.8 & 80.0 / 0.6 & 87.2 / 0.4 & 62.7 / 0.0 & 91.9 / 6.6 & 89.3 / 0.0 & 83.7 / 1.7 \\
\bottomrule
\end{tabular}

\end{sidewaystable}


\begin{sidewaystable}
\caption{AUROC Performance by Method and ID Dataset For Supervised Constrastive Learning Trained Resnet50}


\label{tab:auroc_results_simclr}
\begin{tabular}{llllllllllll}
\toprule
 & Butterfly & Cards & Colon & Eurosat & Fashion & Food & Garbage & Plant & Rock & Tissue & Yoga \\
method &  &  &  &  &  &  &  &  &  &  &  \\
\midrule
KNN & 75.3 / 85.4 & 70.9 / 91.6 & 93.4 / 93.7 & 92.1 / 94.8 & 83.2 / 92.3 & 72.9 / 83.4 & 59.3 / 72.3 & 93.1 / 99.1 & 51.0 / 60.4 & 55.6 / 74.3 & 74.6 / 85.2 \\
DF + KNN & 65.8 / 99.7 & 69.6 / 97.6 & 91.7 / 99.9 & 88.0 / 99.7 & 83.9 / 99.1 & 72.9 / 99.2 & 58.2 / 99.3 & 92.2 / 99.9 & 49.3 / 79.2 & 57.2 / 99.8 & 74.1 / 96.3 \\
Mahalanobis & 83.5 / 90.9 & 73.3 / 93.0 & 89.7 / 80.6 & 94.3 / 95.2 & 79.8 / 86.2 & 70.8 / 81.8 & 67.2 / 84.4 & 90.6 / 96.6 & 47.0 / 68.0 & 51.3 / 74.8 & 75.4 / 86.0 \\
DF + Mahalanobis & 83.4 / 91.3 & 73.4 / 93.1 & 90.1 / 99.5 & 94.3 / 98.2 & 79.9 / 97.1 & 70.8 / 95.5 & 67.9 / 88.1 & 91.3 / 99.8 & 47.8 / 70.6 & 52.7 / 99.0 & 75.6 / 90.8 \\
PT KNN & 53.0 / 99.9 & 52.8 / 98.8 & 73.0 / 99.9 & 73.4 / 99.9 & 78.4 / 99.2 & 67.5 / 99.6 & 62.4 / 99.9 & 81.4 / 99.9 & 58.2 / 98.0 & 58.7 / 99.9 & 57.8 / 99.3 \\
\bottomrule
\end{tabular}

\end{sidewaystable}



\subsection{Discussion On Percentile for Domain Filtering}

The effectiveness of the domain filter can be negatively impacted if the in domain distribution is wider than desired. In particular, the Rock dataset \citep{rock_data} would often set $t_\rvd \approx 1.78$, compared with the Colon dataset at $t_\rvd \approx 0.47$ and the Food dataset at $t_\rvd \approx 1.08$. By changing $p=0.99\to0.98$, we can reduce $FPR@95=52.5\to27.9 $ for the Rock dataset on out-of-domain OOD detection. However, reducing the percentile $p$ will inevitably result in more false positive rejections for in domain data.

In these situations, it may be more appropriate to investigate why the initial assumptions do not hold. Namely, we may want to consider whether or not our dataset is truly a narrow domain dataset and whether or not outliers within the ID data may have an larger than expected influence on the calculation of $t_\rvd$.

A comparison table of domain filtering at different percentiles for the rock dataset can be found in Table \ref{tab:threshold} and the average for all datasets excluding rock can be found in Table \ref{tab:threshold2}. Sample images from the Rock dataset are shown in Figure \ref{fig:rock}, which shows that these images can contain close up shots of rock patterns, but also rock formations in the wild. Interestingly, the dataset creators decided to include what appears to be a marble counter top as a member of the marble class.

\label{percentile}

\begin{sidewaystable}

\caption{Performance for the Rock Dataset across different OOD methods and training methods, considering differing percentiles for the domain filter.  DF98 means domain filter at the 98th percentile and DF99.9 means domain filter at the 99.9th percentile. The default domain filter used in this paper is at the 99th percentile.  }
\label{tab:threshold}

\begin{tabular}{lllllll}
\toprule
 & \multicolumn{3}{r}{FPR@95} & \multicolumn{3}{r}{AUROC} \\
 & CE DinoV2 & CE Resnet & SC Resnet & CE DinoV2 & CE Resnet & SC Resnet \\
method &  &  &  &  &  &  \\
\midrule
PT KNN & 91.9 / 6.6 & 91.9 / 6.6 & 91.9 / 6.6 & 58.2 / 98.0 & 58.2 / 98.0 & 58.2 / 98.0 \\
MSP & 87.2 / 87.0 & 85.8 / 71.8 & NA & 61.3 / 53.6 & 64.2 / 66.2 & NA \\
Energy & 87.3 / 85.8 & 86.7 / 71.2 & NA & 66.5 / 52.3 & 64.0 / 68.6 & NA \\
Mahalanobis & 90.8 / 62.6 & 83.1 / 44.2 & 97.0 / 72.1 & 58.4 / 74.2 & 63.7 / 85.1 & 47.0 / 68.0 \\
NCI & 85.6 / 76.7 & 75.9 / 64.0 & NA & 64.8 / 62.5 & 67.4 / 71.9 & NA \\
KNN & 90.0 / 72.1 & 77.3 / 61.8 & 88.3 / 84.0 & 57.8 / 66.0 & 65.0 / 73.8 & 51.0 / 60.4 \\
ReAct & 87.3 / 84.5 & 84.7 / 75.0 & NA & 66.9 / 53.9 & 63.8 / 65.2 & NA \\
DF + KNN & 91.5 / 57.9 & 75.1 / 52.5 & 87.3 / 71.0 & 56.5 / 82.8 & 64.2 / 84.5 & 49.3 / 79.2 \\
DF + ReAct & 87.3 / 74.8 & 84.9 / 61.0 & NA & 65.7 / 75.0 & 63.1 / 80.3 & NA \\
DF + MDS & 86.8 / 50.5 & 82.0 / 39.8 & 96.5 / 64.6 & 60.1 / 82.9 & 63.0 / 87.4 & 47.8 / 70.6 \\
DF98 + KNN & 90.4 / 34.7 & 76.1 / 27.9 & 94.3 / 44.5 & 56.2 / 92.2 & 63.5 / 93.7 & 47.0 / 90.9 \\
DF99.9 + KNN & 91.9 / 71.1 & 76.0 / 60.1 & 87.9 / 82.2 & 56.7 / 71.5 & 64.5 / 77.2 & 50.1 / 66.0 \\
\bottomrule
\end{tabular}
    
\end{sidewaystable}

\begin{sidewaystable}

\caption{Performance Average for all Dataset (Excluding Rock) across different OOD methods and training methods, considering differing percentiles for the domain filter.  DF98 means domain filter at the 98th percentile and DF99.9 means domain filter at the 99.9th percentile. The default domain filter used in this paper is at the 99th percentile.  }
\label{tab:threshold2}
\begin{tabular}{lllllll}
\toprule
 & \multicolumn{3}{r}{FPR@95} & \multicolumn{3}{r}{AUROC} \\
 & CE DinoV2 & CE Resnet & SC Resnet & CE DinoV2 & CE Resnet & SC Resnet \\
method &  &  &  &  &  &  \\
\midrule
PT KNN & 79.7 / 0.9 & 79.7 / 0.9 & 79.7 / 0.9 & 65.1 / 99.6 & 65.1 / 99.6 & 65.1 / 99.6 \\
MSP & 65.4 / 43.0 & 61.8 / 38.9 & NA & 75.1 / 82.0 & 78.3 / 87.4 & NA \\
Energy & 65.0 / 37.3 & 65.3 / 41.4 & NA & 75.3 / 85.6 & 78.0 / 87.6 & NA \\
Mahalanobis & 62.5 / 18.5 & 59.9 / 16.2 & 62.3 / 34.7 & 75.9 / 93.4 & 78.4 / 94.4 & 78.9 / 87.6 \\
NCI & 66.7 / 35.3 & 74.5 / 36.1 & NA & 74.1 / 86.6 & 73.3 / 88.5 & NA \\
KNN & 61.9 / 25.4 & 64.4 / 25.8 & 61.5 / 32.9 & 75.8 / 91.0 & 76.1 / 91.1 & 78.0 / 87.8 \\
ReAct & 64.2 / 36.4 & 71.9 / 47.7 & NA & 75.9 / 86.3 & 74.4 / 84.9 & NA \\
DF + KNN & 65.2 / 3.2 & 64.3 / 2.5 & 63.8 / 3.2 & 73.9 / 99.0 & 75.8 / 99.2 & 76.2 / 99.0 \\
DF + ReAct & 64.3 / 3.7 & 72.3 / 4.1 & NA & 75.7 / 98.8 & 73.8 / 99.1 & NA \\
DF + MDS & 62.1 / 2.9 & 60.0 / 11.8 & 62.4 / 11.6 & 76.1 / 99.0 & 78.6 / 96.3 & 78.1 / 95.2 \\
DF98 + KNN & 64.8 / 2.4 & 63.8 / 2.2 & 62.7 / 3.3 & 74.0 / 99.0 & 75.8 / 99.2 & 77.1 / 98.9 \\
DF99.9 + KNN & 63.6 / 9.0 & 64.8 / 7.7 & 63.2 / 7.7 & 75.2 / 97.5 & 75.9 / 98.0 & 77.5 / 98.0 \\
\bottomrule
\end{tabular}
    
\end{sidewaystable}

\subsection{Variance Analysis}

\label{variance}

We use a Wilcoxon Signed Rank test to determine the whether or not the improvement offered by domain filtering is statistically consistent, in Table \ref{tab:var}. Due to the non normal distribution of FPR@95 values across seeds, we use a non parametric test, as opposed to the paired t test. We show that for each ID dataset, the average far (out of domain) OOD performance improves with domain filtering in each seed. This is a consistent result implying the domain filtering never harms far (out of domain) OOD performance.


\begin{sidewaystable}
    

\caption{Wilcoxon signed-rank test p-values comparing FPR@95 in Far (Out of Domain) OOD detection using regular KNN and domain filtered KNN across models and ID datasets. Low p-values indicate statistically significant improvements from filtering. Note that the p value of 0.03125 indicates that in all 5 seeds, the domain filtered KNN method achieved lower FPR@95 than the regular KNN method. This is also the lowest possible p value with a paired sample size of 5.}
\label{tab:var}
\begin{tabular}{llllllllllll}
\toprule
 & Tissue & Plant & Yoga & Colon & Garbage & Food & Fashion & Rock & Eurosat & Cards & Butterfly \\
\midrule
CE DinoV2 & 0.03 & 0.03 & 0.03 & 0.03 & 0.03 & 0.03 & 0.03 & 0.03 & 0.03 & 0.03 & 0.03 \\
CE Resnet & 0.03 & 0.03 & 0.03 & 0.03 & 0.03 & 0.03 & 0.03 & 0.03 & 0.03 & 0.03 & 0.03 \\
SC Resnet & 0.03 & 0.03 & 0.03 & 0.03 & 0.03 & 0.03 & 0.03 & 0.03 & 0.03 & 0.03 & 0.03 \\
\bottomrule
\end{tabular}
\end{sidewaystable}



\section{Single Domain Dataset Details}

\label{dataset}


\subsection{Butterfly}  This is a dataset hosted by Kaggle originating from \citep{AIPlanet_DataSprint107_2024}. It consists of 75 classes of Butterflies. It contains 2786 images. See Figure \ref{fig:butterfly} for sample images.

\begin{figure}[htbp]
    \centering
    \includegraphics[width=\columnwidth]{butterfly_1_sample.jpg}
    \caption{Sample images for the Butterfly dataset.}
    \label{fig:butterfly}
\end{figure}

\subsection{Cards} This is a playing card classification dataset by rank and suit \citep{card_data}. This dataset is hosted on Kaggle and consists of 7624 images split into 53 categories. See Figure \ref{fig:cards}.

\begin{figure}[htbp]
    \centering
    \includegraphics[width=\columnwidth]{cards_1_sample.jpg}
    \caption{Sample images for the Cards dataset.}
    \label{fig:cards}
\end{figure}


\subsection{Colon} This is a colon pathology dataset with different diseases labeled \citep{yang2023medmnist}. This dataset consists of 89,996 images in 9 different classes of colon disease. See Figure \ref{fig:colon}.


\begin{figure}[htbp]
    \centering
    \includegraphics[width=\columnwidth]{colon_1_sample.jpg}
    \caption{Sample images for the Colon dataset.}
    \label{fig:colon}
\end{figure}

\subsection{Eurosat} This is a satellite images dataset for classifying different types of land use \citep{helber2019eurosat}. It contains 27,000 labeled images and only RGB images were used from the dataset. See Figure \ref{fig:eurosat}.

\begin{figure}[htbp]
    \centering
    \includegraphics[width=\columnwidth]{eurosat_1_sample.jpg}
    \caption{Sample images for the Eurosat dataset.}
    \label{fig:eurosat}
\end{figure}


\subsection{Fashion} The FashionMNIST dataset describing different articles of clothing \citep{fashion}. It consists of 70,000 grey scale images labeld into 10 classes. See Figure \ref{fig:fashion}.


\begin{figure}[htbp]
    \centering
    \includegraphics[width=\columnwidth]{fashion_1_sample.jpg}
    \caption{Sample images for the Fashion dataset.}
    \label{fig:fashion}
\end{figure}

\subsection{Food} The Food101 dataset \citep{food} contains $101$ classes of different types of food. It consists of 101,000 images with 1000 images per class.

See figure \ref{fig:food}.

\begin{figure}[htbp]
    \centering
    \includegraphics[width=\columnwidth]{food_1_sample.jpg}
    \caption{Sample images for the Food dataset.}
    \label{fig:food}
\end{figure}

\subsection{Garbage} This is a dataset to classify the material of different waste objects \citep{single2023realwaste}. The dataset is split into 9 classes with more than 4000 images and at least 300 images per class. See Figure \ref{fig:garbage}.

\begin{figure}[htbp]
    \centering
    \includegraphics[width=\columnwidth]{garbage_1_sample.jpg}
    \caption{Sample images for the Garbage dataset.}
    \label{fig:garbage}
\end{figure}

\subsection{Plant} This is a plant leaves dataset detailing different types of disease \citep{plant}. There are over 50,000 images across 38 classes. Each variety of plant contains a set of healthy leaf images and one more diseased leaf images. See Figure \ref{fig:plant}.

\begin{figure}[htbp]
    \centering
    \includegraphics[width=\columnwidth]{plant_1_sample.jpg}
    \caption{Sample images for the Plant dataset.}
    \label{fig:plant}
\end{figure}

\subsection{Rock} This is a dataset of different types of rocks and minerals \citep{rock_data}. It consists of 7 different classes across more than 2000 images. See Figure \ref{fig:rock}.

\begin{figure}[htbp]
    \centering
    \includegraphics[width=\columnwidth]{rock_1_sample.jpg}
    \caption{Sample images for the Rock dataset. Note that this dataset appears to contain images in multiple domains, such as the kitchen image of a marble countertop.}
    \label{fig:rock}
\end{figure}



\subsection{Tissue} This is a kidney cortex microscope dataset with various types of tissue labeled \citep{yang2023medmnist}. It consists of over 200,000 images across 8 different classes. See Figure \ref{fig:tissue}

\begin{figure}[htbp]
    \centering
    \includegraphics[width=\columnwidth]{tissue_1_sample.jpg}
    \caption{Sample images for the Tissue dataset.}
    \label{fig:tissue}
\end{figure}



\subsection{Yoga} This is a dataset of people performing different yoga poses from the internet \citep{yoga_data}. It consists of 2,964 images across 6 classes. See Figure \ref{fig:yoga}.

\begin{figure}[htbp]
    \centering
    \includegraphics[width=\columnwidth]{yoga_1_sample.jpg}
    \caption{Sample images for the Yoga dataset.}
    \label{fig:yoga}
\end{figure}

